\chapter*{Acknowledgements}
\label{sec:acknowledgement}\vspace*{-10mm}

\pdfbookmark[0]{Acknowledgements}{Acknowledgements}

\bigskip
\bigskip
\bigskip

Projects like this one are never truly finished and this dissertation
represents a significant milestone on the road to the stories, books and
films I intend to spawn. The fullest realization of this project is our
emancipation, not just words, and I hope this work helps inspire
meaningful actions. There were many times I doubted I would complete
this phase and I am truly ecstatic to be writing these acknowledgments
right now. I could never have arrived at this point without the
heartfelt generosity and loving support of all the people I am about to
thank, and many, many more who conspired, along with the universe, to
help me follow through.

First, I want to thank the Columbia Journalism School's Communications
PhD program, a very special program that carefully nurtures a unique
space for academic dreamers. I was drawn the Communications PhD program
while I was enrolled in Teachers College. As a student in Frank Moretti
and Robbie McClintock's History and Theories of Communication seminar I
first met some of the Comm students and immediately felt a deep kinship,
realizing I had finally found my academic home. The program's faculty
are also remarkable, and each of them opened my mind to entirely new
worlds.

This monograph has benefited greatly from the thoughtful attention and
constructive comments of my dissertation committee: Michael Schudson,
Todd Gitlin, Diane Vaughan, Sayantani DasGupta and Jack Bratich. As I
continue to develop this work your ideas and ideals will help guide me.
Your pragmatic commitment to clear, concise, and compelling prose has
shaped my writing, and ultimately, my thinking.

I want to especially thank Michael and Todd, who demonstrated their
faith in my academic success from the beginning, and helped me develop
as a better writer, thinker, and person. Michael always entertained my
intellectual flights, all the while providing meticulous, thoughtful
feedback. He encouraged me to doggedly pursue my interests, and gently
reminded me to flesh out my conceptual abstractions with more
sociological meat. When I first started working with Todd, he reminded
me of Peter Parker's boss J. Jonah Jameson, a hard-boiled newspaper
editor helping me sniff out the best stories, challenging my conspiracy
theories, and forcing me to back up my hunches with evidence. He became
more than an editor --- he became my advisor, my co-investigator and
finally, my mentor.

In the year leading up to my defense I had the privilege of
participating in a dissertation writing group, and the brilliant
feedback of Burcu Baykurt, Travis Mushett, Ri Pierce-Grove and Soomin
Seo provided me with the inspiration and resolve to see this project
through. Honorary mentions go to Sascha DuBrul, Erica Fletcher, Brad
Lewis, Faith Rhyne and Tom Schmatzhagen and Alexander Gail Shermansong,
close friends from outside my program who also closely read and debated
with me throughout drafts and droughts. Together they helped me weather
the cold winters of endless revisions through the warmth of their dark
humor and good cheer.

To all of my friends, activists and researchers involved in the mad
movement, thank you for sharing your stories and lives with me. I
especially want to thank the membership of the The Icarus Project, the
national collective, and the local NYC Icarus chapter for their vision,
courage and resilience. The perseverance of Mindfreedom, the leadership
of Mad In America, the entire crew of Mindful Occupation and everyone in
the Occupy Wall Street Safety Cluster. For teaching me about mad
languages of compassion Sonia J. Cheruvillil, Will Hall, Rachel Liebert,
Jacks McNamara, Maryse Mitchell-Brody, and Bonfire Madigan Shive. For
mad visions Daniel Mackler, Andrew Grant, Anniken Hoel, Sarah Quinter,
Ken Paul Rosenthal and Mitch McCabe. For mad NYC support Emily Allan,
Caitlin Belforti, Nadia Gomez, Christine Guryev, Erin Lemkey, Kevin
Marx, Jazmine Russel, Becca Stabile, Suki Valentine, Curt White and
Hollie Zegman. For teaching me about mad heroism David Oaks, Jim
Gottstein, David Egilman and Fred von Lohmann. And for teaching me about
mad love, Sascha Altman DuBrul, Faith Rhyne, Eric Stiens, Erica
Fletcher, and Bradley Lewis.

Although digital files seem immaterial, they still require physical
resources to create and sustain them. This dissertation was written in
fits and starts, with regular stints in neighborhood cafes, occasional
bursts in Avery or Burke library, and punctuated by intense writing
retreats hosted by some of my dearest friends who opened their porches,
decks, and hammocks to my feverish pecking. Special thanks to the
Frankfurts for providing Fair Harbor, the Garfields for the joys of
writing while Snowbound, the Fletchers for Mission Control, and the
Mushett-Johnsons for waiving the toll.

As a part-time student I have gotten to know many students and faculty
at Columbia over the years, and want to sincerely thank everyone who has
indulged me in a late-night conversations, read any of my mad papers or
posts, and traveled with me across metaphysical landscapes. The list is
surely truncated, but I can distinctly recall learning essential ideas
which informed this project, even if our conversations were never
referenced in APA format.

To some of my most influential and inspirational teachers: John
Broughton, Mary Marshall-Clark, Nick Mirzeoff, Eben Moglen, Eli Noam,
David Stark, and especially, Robbie McClintock and Frank Moretti.

Some of the amazing faculty and cohorts I have had the privilege to work
with during my decade of part-time enrollment: Chris W. Anderson, Lynn
Berger, Charles Berret, Lluis de Nadal, Andi Dixon, Rosalind Donald,
Laura Forlano, Maxwell Foxman, Lucas Graves, Tucker Harding, Richard
John, Joselyn Jurich, John Kelly, Kathryn Montalbano, Rasmus Neilsen,
Ruthie Palmer, Benjamin Peters, Julia Sonnevend, Phil Stephenson, Madiha
Tahir, Andie Tucher, Joost van Dreunen, Steve Welsh.

The tri-state communications community: Gus Andrews, Peter Assaro,
Biella Coleman, Matt Curinga, Christina Hester-Dunbar, Alex Gil, Alissa
Quart, Nathan Schneider, Trebor Scholz, Aram Sinnreich and Dennis Tenen.

And some of my closest friends, fans and cheerleaders: Anita Altman,
Micah Anderson, Alina Antoniou, Dan Beeby, Mar Cabra, Liz Day, Sky and
Maasha Duveen, Rena Chicklas, Laura Chzaszcz, Jed and Lilly Davis,
Michelle Flick, John and Lilliana Frankfurt, Rob Garfield, Teresa
Gonzalez, Seth Grossman, Aziz and Marisa Isham, Adam and Adina Kahn,
Peter B. Kaufman, Tsering Lama, Maurice Matiz, Karla Myerson, Mark
Phillipson, Michael Preston, Niral Shah, Robin Stern, Melissa Mannis,
Avery Rosen, Ian Sullivan, Corinne Tinacci, the whole crews of
Blunderbuss Magazine and the Sacred Spaces camp.

Finally, I turn to the mishpacha, where the blood has always run thick.
I literally would not exist without my parents, Ken and Anne Bossewitch.
Their commitment to the quality of my education and support through the
calm and the turbulent has gifted me with the privilege of completing
this phase of my life's project. Special thanks to my step-parents,
Marty Hochberg and Golda Pearlman for preserving our family traditions
-- you both fit right in (an exquisite compliment!). Marty was
especially critical in helping me resolve my thorniest philosophical
conundrums. Uncle Joel, Aunt Margie and Nili (and family) have also
accompanied me throughout my life journey, offering me love, kindness,
and delicious family meals throughout. To my siblings and their
respective clans: Tammy, Tsuri, Yitav, Yuval, Yakir, Yedidyah and Maayan
-- enjoy the light shabbat reading. Avi, Shooley, Leora, Tova, Moshe,
Avrum and Malka, bearers of the Bossewitch torch -- you're up next ;-).

All of my readers helped bring out the best in my work, often providing
me with challenging revisions, which I grudgingly had to agree with.
Grudging, since they often entailed a great deal of work, but ultimately
led to this monograph. They helped me balanced argument, one that is
intended to reach beyond the choir. And most of all, he supported me in
the authoring of a truly interdisciplinary work - one that is accessible
beyond the echo chambers of academic fiefdoms.
